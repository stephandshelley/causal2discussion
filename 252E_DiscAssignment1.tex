\documentclass[ignorenonframetext,]{beamer}
\setbeamertemplate{caption}[numbered]
\setbeamertemplate{caption label separator}{: }
\setbeamercolor{caption name}{fg=normal text.fg}
\beamertemplatenavigationsymbolsempty
\usepackage{lmodern}
\usepackage{amssymb,amsmath}
\usepackage{ifxetex,ifluatex}
\usepackage{fixltx2e} % provides \textsubscript
\ifnum 0\ifxetex 1\fi\ifluatex 1\fi=0 % if pdftex
  \usepackage[T1]{fontenc}
  \usepackage[utf8]{inputenc}
\else % if luatex or xelatex
  \ifxetex
    \usepackage{mathspec}
  \else
    \usepackage{fontspec}
  \fi
  \defaultfontfeatures{Ligatures=TeX,Scale=MatchLowercase}
\fi
% use upquote if available, for straight quotes in verbatim environments
\IfFileExists{upquote.sty}{\usepackage{upquote}}{}
% use microtype if available
\IfFileExists{microtype.sty}{%
\usepackage{microtype}
\UseMicrotypeSet[protrusion]{basicmath} % disable protrusion for tt fonts
}{}
\newif\ifbibliography
\hypersetup{
            pdftitle={Discussion Assignment \#1},
            pdfauthor={Sabrina Boyce, Shelley Facente, and Steph Holm},
            pdfborder={0 0 0},
            breaklinks=true}
\urlstyle{same}  % don't use monospace font for urls

% Prevent slide breaks in the middle of a paragraph:
\widowpenalties 1 10000
\raggedbottom

\AtBeginPart{
  \let\insertpartnumber\relax
  \let\partname\relax
  \frame{\partpage}
}
\AtBeginSection{
  \ifbibliography
  \else
    \let\insertsectionnumber\relax
    \let\sectionname\relax
    \frame{\sectionpage}
  \fi
}
\AtBeginSubsection{
  \let\insertsubsectionnumber\relax
  \let\subsectionname\relax
  \frame{\subsectionpage}
}

\setlength{\parindent}{0pt}
\setlength{\parskip}{6pt plus 2pt minus 1pt}
\setlength{\emergencystretch}{3em}  % prevent overfull lines
\providecommand{\tightlist}{%
  \setlength{\itemsep}{0pt}\setlength{\parskip}{0pt}}
\setcounter{secnumdepth}{0}

\title{Discussion Assignment \#1}
\author{Sabrina Boyce, Shelley Facente, and Steph Holm}
\date{10/7/2019}

\begin{document}
\frame{\titlepage}

\section{Question 8: Give a detailed implementation of longitudinal IPTW
to estimate parameters of an MSM without effect modifiers (Section
7-8).}\label{question-8-give-a-detailed-implementation-of-longitudinal-iptw-to-estimate-parameters-of-an-msm-without-effect-modifiers-section-7-8.}

\begin{frame}{Implementing the Horvitz-Thompson Estimator}

We are interested in the expected \(Y\) if everyone got treatment regime
\(\bar{A}(t) = \bar{a}\), for t= 0,1.
\[ \Psi^F(P_{U,X}) = E_{U,X}[Y_{\bar{A}(t)=\bar{a}}] \]

The Horvitz-Thompson IPTW estimator for \(\Psi^F\left(P_{U,X}\right)\)
is:

\begin{align*}
\hat{\Psi}(P_n) & =\frac{1}{n}\sum_{i=1}^n\frac{\mathbb{I}[\bar{A}_i(t)=\bar{a}]}{g_n(A_i(0)|L_i(0))\times g_n(A_i(1)|A_i(0), L_i(0), L_i(1))}Y_i\\
\end{align*}

\end{frame}

\begin{frame}{Step 1: Calculate appropriate stabilized weights using the
modified Horvitz-Thomas estimator}

\(Weights = \frac{1}{g_n(A_i(0) | L_i(0))\times g_n(A_i(1) | A_i(0), L_i(0), L_i(1))}\)

\vspace{6 mm}

\begin{block}{Part A:}

Estimate the probability of receiving treatment using correctly
specified parametric regression models (logistic regression)

\begin{align*}
g_0(A(0)=a(0)|L(0)) &= expit[\beta_0 + \beta_1 L(0)] \\
g_0(A(1)=a(1)|\bar{L}(1), A(0)) &= expit[\beta_0 + \beta_1 L(0) + \beta_2 L(1)] \\
\end{align*}

\textbf{In this example} we are estimating the probability of being
treated with AZT at each time point, given the covariate pattern and the
prior history of AZT treatment.

\end{block}

\end{frame}

\begin{frame}{Step 1: Calculate appropriate stabilized weights using the
modified Horvitz-Thomas estimator}

\begin{block}{Part B:}

Predict each subject's probability of the exposure at each time t, given
his or her observed exposure and covariate history.

\vspace{6 mm}

\(g_n(A_i(t)=a_i(t) | \bar{A}_i(t-1), \bar{L_i}(t))\)

\vspace{6mm}

\textbf{In this example:}

\begin{itemize}
\tightlist
\item
  for time points where AZT treatment is NOT occuring it is the
  predicted probability of NOT being treated, given the observed past.
\item
  for time points where AZT treatment IS occuring it is the predicted
  probability of being treated, given the observed past.
\end{itemize}

\end{block}

\end{frame}

\begin{frame}{Step 1: Calculate appropriate stabilized weights using the
modified Horvitz-Thomas estimator}

\begin{block}{Part C:}

Predict each subject's probability of the entire exposure history, which
is the product of the time point specific probabilities.

\vspace{6mm}

\(\prod_{t=1}^k(A_i(t) | \bar{A_i}(t-1),\bar{L_i}(t))\)

\vspace{6mm}

\textbf{In this example} we are estimating the probability of their
entire AZT treatment hitory pattern.

\vspace{4mm}

The weights, as given earlier, are thus the inverse of these products.

\end{block}

\end{frame}

\begin{frame}{Step 2: Take the weighted average of observed outcomes
across the population}

\vspace{3mm}

The Horvitz-Thompson IPTW estimator for \(\Psi^F\left(P_{U,X}\right)\)
is:

\begin{align*}
\hat{\Psi}(P_n) & =\frac{1}{n}\sum_{i=1}^n\frac{\mathbb{I}[\bar{A}_i(t)=\bar{a}]}{g_n(A_i(0)|L_i(0))\times g_n(A_i(1)|A_i(0), L_i(0), L_i(1))}Y_i\\
\end{align*}

The Modified or Stabilized Horvitz-Thompson IPTW estimator for
\(Psi^F\left(P_{U,X}\right)\) is:

\begin{align*}
\hat{\Psi}(P_n) & =\dfrac{\frac{1}{n}\sum_{i=1}^n\frac{\mathbb{I}[\bar{A}_i(t)=\bar{a}]}{g_n(A_i(0)|L_i(0))\times g_n(A_i(1)|A_i(0), L_i(0), L_i(1))}Y_i}{\frac{1}{n}\sum_{i=1}^n\frac{\mathbb{I}[\bar{A}_i(t)=\bar{a}]}{g_n(A_i(0)|L_i(0))\times g_n(A_i(1)|A_i(0), L_i(0), L_i(1))}}\\
\end{align*}

\end{frame}

\section{Question 8: Give a detailed implementation of longitudinal IPTW
to estimate parameters of an MSM without effect modifiers (Section
7-8).}\label{question-8-give-a-detailed-implementation-of-longitudinal-iptw-to-estimate-parameters-of-an-msm-without-effect-modifiers-section-7-8.-1}

\begin{frame}{Implementing the Estimator}

We are interested in the expected \(Y\) if everyone got treatment regime
\(\bar{A}(t) = \bar{a}\), for t= 0,\ldots{},k.
\[ \Psi^F(P_{U,X}) = E_{U,X}[Y_{\bar{A}(t)=\bar{a}}] \]

The stabilized weights are:

\(sw_i & = \frac{\prod_{k=0}^K g_n(A_k= a_{ki} | \bar{A}_{k-1} = \bar{a}_{(k-1)i}{\prod_{k=0}^K g_n(A_k= a_{ki} | \bar{A}_{k-1} = \bar{a}_{(k-1)i}, \bar{L}_k = \bar{l}_{ki}}\)

\end{frame}

\begin{frame}{Step 1: Calculate appropriate stabilized weights using the
modified Horvitz-Thomas estimator}

\(Weights = \frac{1}{g_n(A_i(0) | L_i(0))\times g_n(A_i(1) | A_i(0), L_i(0), L_i(1))}\)

\vspace{6 mm}

\begin{block}{Part A:}

Estimate the probability of receiving treatment using correctly
specified parametric regression models (logistic regression)

\begin{align*}
g_0(A(0)=a(0)|L(0)) &= expit[\beta_0 + \beta_1 L(0)] \\
g_0(A(1)=a(1)|\bar{L}(1), A(0)) &= expit[\beta_0 + \beta_1 L(0) + \beta_2 L(1)] \\
\end{align*}

\textbf{In this example} we are estimating the probability of being
treated with AZT at each time point, given the covariate pattern and the
prior history of AZT treatment.

\end{block}

\end{frame}

\begin{frame}{Step 1: Calculate appropriate stabilized weights using the
modified Horvitz-Thomas estimator}

\begin{block}{Part B:}

Predict each subject's probability of the exposure at each time t, given
his or her observed exposure and covariate history.

\vspace{6 mm}

\(g_n(A_i(t)=a_i(t) | \bar{A}_i(t-1), \bar{L_i}(t))\)

\vspace{6mm}

\textbf{In this example:}

\begin{itemize}
\tightlist
\item
  for time points where AZT treatment is NOT occuring it is the
  predicted probability of NOT being treated, given the observed past.
\item
  for time points where AZT treatment IS occuring it is the predicted
  probability of being treated, given the observed past.
\end{itemize}

\end{block}

\end{frame}

\begin{frame}{Step 1: Calculate appropriate stabilized weights using the
modified Horvitz-Thomas estimator}

\begin{block}{Part C:}

Predict each subject's probability of the entire exposure history, which
is the product of the time point specific probabilities.

\vspace{6mm}

\(\prod_{t=1}^k(A_i(t) | \bar{A_i}(t-1),\bar{L_i}(t))\)

\vspace{6mm}

\textbf{In this example} we are estimating the probability of their
entire AZT treatment hitory pattern.

\vspace{4mm}

The weights, as given earlier, are thus the inverse of these products.

\end{block}

\end{frame}

\begin{frame}{Step 2: Take the weighted average of observed outcomes
across the population}

\vspace{3mm}

The Horvitz-Thompson IPTW estimator for \(\Psi^F\left(P_{U,X}\right)\)
is:

\begin{align*}
\hat{\Psi}(P_n) & =\frac{1}{n}\sum_{i=1}^n\frac{\mathbb{I}[\bar{A}_i(t)=\bar{a}]}{g_n(A_i(0)|L_i(0))\times g_n(A_i(1)|A_i(0), L_i(0), L_i(1))}Y_i\\
\end{align*}

The Modified or Stabilized Horvitz-Thompson IPTW estimator for
\(Psi^F\left(P_{U,X}\right)\) is:

\begin{align*}
\hat{\Psi}(P_n) & =\dfrac{\frac{1}{n}\sum_{i=1}^n\frac{\mathbb{I}[\bar{A}_i(t)=\bar{a}]}{g_n(A_i(0)|L_i(0))\times g_n(A_i(1)|A_i(0), L_i(0), L_i(1))}Y_i}{\frac{1}{n}\sum_{i=1}^n\frac{\mathbb{I}[\bar{A}_i(t)=\bar{a}]}{g_n(A_i(0)|L_i(0))\times g_n(A_i(1)|A_i(0), L_i(0), L_i(1))}}\\
\end{align*}

\end{frame}

\section{Question 9: How would you modify the above procedure when the
target causal parameter is a MSM with effect modification (Section
9)?}\label{question-9-how-would-you-modify-the-above-procedure-when-the-target-causal-parameter-is-a-msm-with-effect-modification-section-9}

\begin{frame}{Including Baseline Covariates in the Model}

\vspace{3mm} If effect modification by baseline covariates \(V\) (a
subset of \(L(t)\)) is of interest to the target causal parameter,
inclusion of those baseline characteristics in the MSM allows for their
incorporation into the counterfactual pseudopopulations.

\vspace{3mm} Therefore, we want to condition on baseline covariates
\(V\) in our model, where the model is now: \vspace{-6mm}

\begin{align*}
E[Y_{\bar{a}}|V] = m(\bar{a},V|\beta) = \beta_0 + \beta_1\sum_{t=0}^1a(t) + \beta_2V + \beta_3\sum_{t=0}^1a(t) \times V
\end{align*}

\vspace{3mm} The choice of numerator in the MSM changes the target
parameter being measured, and the stabilized weights can be improved:
\vspace{-5mm}

\begin{align*}
\hat{sw}_\mathnormal{i} = \frac{g_n(\bar{A}_\mathnormal{i}(1)|V_i)}{\prod_{t=0}^1g_n(A_\mathnormal{i}(t)|\bar{A}(t-1), \bar{L}_\mathnormal{i}(t))}\\
\end{align*}

\end{frame}

\end{document}
