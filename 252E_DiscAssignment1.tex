\documentclass[]{article}
\usepackage{lmodern}
\usepackage{amssymb,amsmath}
\usepackage{ifxetex,ifluatex}
\usepackage{fixltx2e} % provides \textsubscript
\ifnum 0\ifxetex 1\fi\ifluatex 1\fi=0 % if pdftex
  \usepackage[T1]{fontenc}
  \usepackage[utf8]{inputenc}
\else % if luatex or xelatex
  \ifxetex
    \usepackage{mathspec}
  \else
    \usepackage{fontspec}
  \fi
  \defaultfontfeatures{Ligatures=TeX,Scale=MatchLowercase}
\fi
% use upquote if available, for straight quotes in verbatim environments
\IfFileExists{upquote.sty}{\usepackage{upquote}}{}
% use microtype if available
\IfFileExists{microtype.sty}{%
\usepackage{microtype}
\UseMicrotypeSet[protrusion]{basicmath} % disable protrusion for tt fonts
}{}
\usepackage[margin=1in]{geometry}
\usepackage{hyperref}
\hypersetup{unicode=true,
            pdftitle={Discussion Assignment \#1},
            pdfauthor={Sabrina Boyce, Shelley Facente, and Steph Holm},
            pdfborder={0 0 0},
            breaklinks=true}
\urlstyle{same}  % don't use monospace font for urls
\usepackage{graphicx,grffile}
\makeatletter
\def\maxwidth{\ifdim\Gin@nat@width>\linewidth\linewidth\else\Gin@nat@width\fi}
\def\maxheight{\ifdim\Gin@nat@height>\textheight\textheight\else\Gin@nat@height\fi}
\makeatother
% Scale images if necessary, so that they will not overflow the page
% margins by default, and it is still possible to overwrite the defaults
% using explicit options in \includegraphics[width, height, ...]{}
\setkeys{Gin}{width=\maxwidth,height=\maxheight,keepaspectratio}
\IfFileExists{parskip.sty}{%
\usepackage{parskip}
}{% else
\setlength{\parindent}{0pt}
\setlength{\parskip}{6pt plus 2pt minus 1pt}
}
\setlength{\emergencystretch}{3em}  % prevent overfull lines
\providecommand{\tightlist}{%
  \setlength{\itemsep}{0pt}\setlength{\parskip}{0pt}}
\setcounter{secnumdepth}{0}
% Redefines (sub)paragraphs to behave more like sections
\ifx\paragraph\undefined\else
\let\oldparagraph\paragraph
\renewcommand{\paragraph}[1]{\oldparagraph{#1}\mbox{}}
\fi
\ifx\subparagraph\undefined\else
\let\oldsubparagraph\subparagraph
\renewcommand{\subparagraph}[1]{\oldsubparagraph{#1}\mbox{}}
\fi

%%% Use protect on footnotes to avoid problems with footnotes in titles
\let\rmarkdownfootnote\footnote%
\def\footnote{\protect\rmarkdownfootnote}

%%% Change title format to be more compact
\usepackage{titling}

% Create subtitle command for use in maketitle
\newcommand{\subtitle}[1]{
  \posttitle{
    \begin{center}\large#1\end{center}
    }
}

\setlength{\droptitle}{-2em}

  \title{Discussion Assignment \#1}
    \pretitle{\vspace{\droptitle}\centering\huge}
  \posttitle{\par}
    \author{Sabrina Boyce, Shelley Facente, and Steph Holm}
    \preauthor{\centering\large\emph}
  \postauthor{\par}
      \predate{\centering\large\emph}
  \postdate{\par}
    \date{10/14/2019}


\begin{document}
\maketitle

\section{Question 1: Specify the question(s) of
interest}\label{question-1-specify-the-questions-of-interest}

How does detectability of HIV RNA in patients' blood vary as a function
of the cumulative AZT dose history?

\section{\texorpdfstring{Question 2: For the AZT example, specify the
longitudinal causal model \(\mathcal{M}^F\). ( See Lecture 1 for help.)
What is the corresponding
DAG?}{Question 2: For the AZT example, specify the longitudinal causal model \textbackslash{}mathcal\{M\}\^{}F. ( See Lecture 1 for help.) What is the corresponding DAG?}}\label{question-2-for-the-azt-example-specify-the-longitudinal-causal-model-mathcalmf.-see-lecture-1-for-help.-what-is-the-corresponding-dag}

*Complete Structural Causal Model

\begin{align*}
O={\bar{L}(k), \bar{A}(k), Y(k+1)}
$\bar_L(k)$ = all measured risk factors for Y (CD4 count, age, white blood count, hematocrit, AIDS diagnosis, symptoms)
$\bar_A(k)$ = AZT dosage (100mg)
Y(k+1) = HIV RNA detected in blood
\end{align*}

\(\bar_L(k)\) = f\_\{L(k)\}(\bar\{U\_\{L(k)\}\}, \bar\{A\}(k-1),
\bar\{L\}(k-1)) \(\bar_A(k)\) = f\_\{A(k)\}(\bar\{U\_\{A(k)\}\},
\bar\{A\}(k-1), \bar\{L\}(k)) Y(k+1) =
f\_\{Y(k+1)\}(\bar\{U\_\{Y(k+1)\}\}, \bar\{A\}(k), \bar\{L\}(k))

The DAG is represented as Figure 1a in the article.

\section{Question 3: Specify the counterfactuals of interest. How are
these counterfactuals generated using an
NPSEM?}\label{question-3-specify-the-counterfactuals-of-interest.-how-are-these-counterfactuals-generated-using-an-npsem}

*complete Our counterfactuals of interest are the values of
\(Y_{\bar{a_k}}\) we would have observed if \(A_0\) had been set to
values 0, 1, 2, 3,\ldots{}15 at each time point, all the 16 possible AZT
doses at each time point. We are able to generate these counterfactuals
by plugging in every combination of treatment level across all time
points and observing the outcome for each.

\section{Question 4: Specify the target causal parameters. Discuss their
interpretation}\label{question-4-specify-the-target-causal-parameters.-discuss-their-interpretation}

*COMPLETE The target causal parameter is given by the following marginal
structural model:

\begin{align}
E(Y_{\bar{a_k}})=m(\bar{a}|\beta= \beta_0 + \beta_1\sum_{t=1}^ka(t)
\end{align}

Our target causal parameter is \(\beta_1\) so that we can calculate
\(e^\beta_1\) which would give us the causal OR for HIV RNA detection
associated with an increasing AZT dose of 100mg. \(\beta_1\) represents
the causal parameter if the relationship between AZT dose and HIV RNA is
unconfounded by measured or unmeasured factors.

\section{\texorpdfstring{Question 5: What are the observed data? What is
the link between the observed data and the structural causal model
\(\mathcal{M}^F\)? Do we place any restrictions on the observed data
model
\(\mathcal{M}\)?}{Question 5: What are the observed data? What is the link between the observed data and the structural causal model \textbackslash{}mathcal\{M\}\^{}F? Do we place any restrictions on the observed data model \textbackslash{}mathcal\{M\}?}}\label{question-5-what-are-the-observed-data-what-is-the-link-between-the-observed-data-and-the-structural-causal-model-mathcalmf-do-we-place-any-restrictions-on-the-observed-data-model-mathcalm}

N i.i.d. copies of \(O=(\bar{A(t)}, \bar{L(t)})~P_0\)

\(\mathcal{M}^F\) contains the distribution of \(\mathcal{M}\)\\
Probability of observing our data (P0) is in the scm

There are no restrictions on the observed data model M.

\section{Question 6: Identifiability and the statistical
estimand}\label{question-6-identifiability-and-the-statistical-estimand}

\subsection{\texorpdfstring{(a) Consider a single time point
intervention. Discuss the assumptions needed to identify the average
treatment effect \(E_{U,X}(Y_1 -Y_0)\) from the observed data
distribution. Discuss the backdoor criteria. (See Lecture Notes from
252D for a
refresher.)}{(a) Consider a single time point intervention. Discuss the assumptions needed to identify the average treatment effect E\_\{U,X\}(Y\_1 -Y\_0) from the observed data distribution. Discuss the backdoor criteria. (See Lecture Notes from 252D for a refresher.)}}\label{a-consider-a-single-time-point-intervention.-discuss-the-assumptions-needed-to-identify-the-average-treatment-effect-e_uxy_1--y_0-from-the-observed-data-distribution.-discuss-the-backdoor-criteria.-see-lecture-notes-from-252d-for-a-refresher.}

In order to identify the average treatment effect from the observed data
distribution in a single point treatment context, we would need to
verify the randomization assumption (that \(\bar{a(t)}|L\) is
independent of Y), the independence assumption (met by conditioning on L
so that U\_A is independent of U\_Y and U\_L is independent of U\_Y),
and the positivity assumption (\min\_\{aEA\}
P(A=a\textbar{}L)\textgreater{}0).

The backdoor criteria are: 1. All spurious sources of association
between A and Y are blocked 2. No new spurious associations are created
(no colliders are conditioned upon) 3. No part of the causal pathway
between A and Y are conditioned upon (we are not conditioning on a
factor on the causal pathway between A and Y).

In this example of a point treatment setting, if we condition on L we
would close all backdoor paths between A and Y without conditioning on a
collider or mediator.

\subsection{\texorpdfstring{(b) Consider a longitudinal setting and the
following DAG. Can you specify a set of covariates that satisfy the
backdoor criteria to identify the expected counterfactual outcome
\(E_{U,X}(Y_{\bar{a}})\) under a joint intervention on A\_0 and
A\_1?}{(b) Consider a longitudinal setting and the following DAG. Can you specify a set of covariates that satisfy the backdoor criteria to identify the expected counterfactual outcome E\_\{U,X\}(Y\_\{\textbackslash{}bar\{a\}\}) under a joint intervention on A\_0 and A\_1?}}\label{b-consider-a-longitudinal-setting-and-the-following-dag.-can-you-specify-a-set-of-covariates-that-satisfy-the-backdoor-criteria-to-identify-the-expected-counterfactual-outcome-e_uxy_bara-under-a-joint-intervention-on-a_0-and-a_1}

No, we do not have a set of covariates to satisfy the backdoor criteria.
If we want to intervene on A(0) and A(1) simultaneously and we try to
condition on L(0) and L(1), we would \ldots{}.If you condition on just
L(1) this happens, and if you do just L(2)\ldots{}.

\subsection{\texorpdfstring{(c) What alternative identifiability
assumptions would be sufficient in this case? What statistical estimand
is equal to \(E_{U,X}(Y_{\bar{a}})\) under this
assumption?}{(c) What alternative identifiability assumptions would be sufficient in this case? What statistical estimand is equal to E\_\{U,X\}(Y\_\{\textbackslash{}bar\{a\}\}) under this assumption?}}\label{c-what-alternative-identifiability-assumptions-would-be-sufficient-in-this-case-what-statistical-estimand-is-equal-to-e_uxy_bara-under-this-assumption}

Siguential backdoor criteria and positivity See pic

\section{Question 7: What is the intuition behind longitudinal
IPTW?}\label{question-7-what-is-the-intuition-behind-longitudinal-iptw}

In general, with IPTW, the idea is that some pattern of covariates can
predict the probability of having received the exposure. But because the
covariates are related to exposure, some covariate-exposure combinations
will be over-represented in the data and some will be under-represented.
By upweighting the contributions of the under-represented exposure
combinations (and vice-versa) we can approximate the data that would
result from a randomized controlled trial. In the longitudinal setting
specifically, the individuals who are upweighted are those whose
\emph{entire} exposure history is rare. An individual's weight is the
inverse of the probability that the subject received the treatment that
they did.

\section{Question 8: Give a detailed implementation of longitudinal IPTW
to estimate parameters of an MSM without effect modifiers (Section
7-8).}\label{question-8-give-a-detailed-implementation-of-longitudinal-iptw-to-estimate-parameters-of-an-msm-without-effect-modifiers-section-7-8.}

\subsection{Implementing the
Estimator}\label{implementing-the-estimator}

The stabilized weights for a marginal structural model are:

\vspace{8mm}

\begin{align*}
sw_i = \frac{\prod_{k=0}^K g_n(A_k= a_{ki})}{\prod_{k=0}^K g_n(A_k= a_{ki} | \bar{A}_{k-1} = \bar{a}_{(k-1)i}, \bar{L}_k = \bar{l}_{ki})}
\end{align*}

\subsection{Step 1: Calculate appropriate stabilized
weights}\label{step-1-calculate-appropriate-stabilized-weights}

\subsubsection{Part A:}\label{part-a}

Estimate the probability of receiving treatment, based on prior
treatment and covariates using correctly specified parametric regression
models (logistic regression) to get the denominator of the weights.

\begin{itemize}
\tightlist
\item
  \textbf{In this example} we are estimating the probability of being
  treated with AZT at each time point, given the covariate pattern and
  the prior history of AZT treatment, and the authors use this model:
\end{itemize}

\begin{align*}
logit (p[A_k = 1| \bar{A}_{k-1} = \bar{a}_{k-1}, \bar{L}_k = \bar{l}_{k})]) &= \alpha_0 + \alpha_1k + \alpha_2a_{k-1} + \alpha_3a_{k-2} \\
& + \alpha_4l_k + \alpha_4l_{k-1} + \alpha_5l_{k-2} + \alpha_6a_{k-1}l_k \\
& + \alpha_7l_0 \\
\end{align*}

Where \(l_k\) is the covariate vector.

\subsection{Step 1: Calculate appropriate stabilized
weights}\label{step-1-calculate-appropriate-stabilized-weights-1}

\subsubsection{Part B:}\label{part-b}

Estimate the probability of receiving treatment, based on prior
treatment (but NOT covariates) to get the numerator of the weights.

\begin{itemize}
\tightlist
\item
  \textbf{In this example} We simply remove those terms from the model
  that depended on l, our covariates, resulting in this model:
\end{itemize}

\begin{align*}
logit p[A_k = 1| \bar{A}_{k-1} = \bar{a}_{k-1}] = \alpha_0^* + \alpha_1^*k + \alpha_2^*a_{k-1} + \alpha_3^*a_{k-2}
\end{align*}

\subsection{Step 1: Calculate appropriate stabilized
weights}\label{step-1-calculate-appropriate-stabilized-weights-2}

\subsubsection{Part C:}\label{part-c}

Predict each subject's probability of the entire exposure history, from
each of the models (based on either ONLY exposure history or exposure
history PLUS covariates). Those predicted values are then plugged into
this equation to get the stabilized weight for each subject, i.

\begin{itemize}
\item
  \textbf{In this example} we are estimating the probability of their
  entire AZT treatment history pattern.
\item
  Say that \(\rho_{ki}\) is the probability of treatment for the ith
  subject at time k given exposure and covariates, but \(\rho_{ki}^*\)
  is the probability of treatment for the ith subject at time k given
  exposure history only. Then each subject's weight is:
\end{itemize}

\begin{align*}
sw_i = \frac{\prod_{k=1}^K(\rho_{ki}^*)^{a_{ki}}(1-\rho_{ki}^*)^{1-a_{ki}}}{\prod_{k=1}^K(\rho_{ki})^{a_{ki}}(1-\rho_{ki})^{1-a_{ki}}}
\end{align*}

\subsection{Step 2: Take the weighted average of observed outcomes
across the
population}\label{step-2-take-the-weighted-average-of-observed-outcomes-across-the-population}

Here we are trying to get our estimate of the \(\beta\) using the
stabilized weights we calculated before. We fit a parsimonious MSM (with
our stablized weights) such as:

\begin{align*}
logit (p[Y=1 | \bar{A} = \bar{a}]) = \beta_0 +\beta_1\sum_{t=0}^Ka(t)
\end{align*}

\(\beta_1\) from this model is then our MSM IPTW estimate.

\section{Question 9: How would you modify the above procedure when the
target causal parameter is a MSM with effect modification (Section
9)?}\label{question-9-how-would-you-modify-the-above-procedure-when-the-target-causal-parameter-is-a-msm-with-effect-modification-section-9}

\subsection{Including Baseline Covariates in the
Model}\label{including-baseline-covariates-in-the-model}

\vspace{3mm} If effect modification by baseline covariates \(V\) (a
subset of \(L(t)\)) is of interest to the target causal parameter,
inclusion of those baseline characteristics in the MSM allows for their
incorporation into the counterfactual pseudo-populations.

\vspace{3mm} Therefore, we want to condition on baseline covariates
\(V\) in our model, where the true model is now: \vspace{-6mm}
\fontsize{10}{20}

\begin{align*}
&logit(Pr[Y_{\bar{a}_k}|V]) = m(\bar{a},V|\beta) = \beta_0 + \beta_1\sum_{t=0}^Ka(t) + \beta_2V + \beta_3\sum_{t=0}^Ka(t) \times V \\
&\text{and the working model is now:} \\
&\beta(P_{U,X}|m) = \underset{\beta}{\operatorname{argmin}} \ E_{U,X}[\sum_{\bar{a} \in \mathcal{A}} \ (Y_{\bar{a}} - m(\bar{a}, V |\beta))^2] \\
\end{align*}

\subsection{Including Baseline Covariates in the
Model}\label{including-baseline-covariates-in-the-model-1}

The choice of numerator in the MSM changes the target parameter being
measured, and the stabilized weights can be improved: \vspace{-5mm}

\begin{align*}
\hat{sw}_\mathnormal{i} = \frac{g_n(\bar{A}_\mathnormal{i}(K)|V_i)}{\prod_{t=1}^Kg_n(A_\mathnormal{i}(t)|\bar{A}(t-1), \bar{L}_\mathnormal{i}(t))}\\
\end{align*}

While we can use a variety of predicted values in the numerator,
ultimately we want the one that is going to make the weights closest to
one (i.e.~the numerator has as many overlapping factors with the
denominator as possible, so that most terms in the fraction cancel out).

\section{\texorpdfstring{Question 10: How does censoring change (a) the
scientific question, (b) the causal model \(\mathcal{M}^F\), (c) the
counterfactuals and target causal parameter, (d) the observed data, (e)
identifiability and (f)
estimation}{Question 10: How does censoring change (a) the scientific question, (b) the causal model \textbackslash{}mathcal\{M\}\^{}F, (c) the counterfactuals and target causal parameter, (d) the observed data, (e) identifiability and (f) estimation}}\label{question-10-how-does-censoring-change-a-the-scientific-question-b-the-causal-model-mathcalmf-c-the-counterfactuals-and-target-causal-parameter-d-the-observed-data-e-identifiability-and-f-estimation}

The article suggests that censoring be considered as an additional
time-varying treatment. This causes the following changes: * the
scientific question: does not change * the causal model: add an
indicator \(C_k\) at each time point, which is an indicator of whether
the outcome was observed at time k, and is equal to 1 if the subject was
censored * the counterfactuals: the counterfactual outcomes must now
include both that the subject followed a particular treatment regimen
(\(\bar{a}\)) and that they were never censored (\(\bar{C}=0\)),so the
counterfactuals are \(Y(\bar{A}(t)= \bar{a}(t), \bar{C}(K)=0\) * the
causal parameter: the same as previous, but with the addition to the
counterfactuals noted above. So the parameter is
\(\psi^F (\mathcal{P}_{U,X})= E_{U,X}(Y_{\bar{a}, \bar{C}=0})\) * the
observed data: presumably if there is censoring some subjects will be
missing data at later time points * identifiability: we must assume that
there are no unmeasured factors in common for \(A_k\) and \(C_k\) for
all values of k, and that censoring is independent of the outcome Y at
all times t. * estimation: it is now an
inverse-probability-of-treatment-and-censoring estimator.

\section{\texorpdfstring{Question 11: In Section 11, the authors note
``our IPTW estimators will be biased and thus MSMs should not be used in
studies in which at each time k there is a covariate level \(l_k\) such
that all subjects with that level of the covariate are certain to
receive the identical treatment \(a_k\)''. What assumption are the
authors referring
to?}{Question 11: In Section 11, the authors note our IPTW estimators will be biased and thus MSMs should not be used in studies in which at each time k there is a covariate level l\_k such that all subjects with that level of the covariate are certain to receive the identical treatment a\_k. What assumption are the authors referring to?}}\label{question-11-in-section-11-the-authors-note-our-iptw-estimators-will-be-biased-and-thus-msms-should-not-be-used-in-studies-in-which-at-each-time-k-there-is-a-covariate-level-l_k-such-that-all-subjects-with-that-level-of-the-covariate-are-certain-to-receive-the-identical-treatment-a_k.-what-assumption-are-the-authors-referring-to}

They are referring to the positivity assumption. In order for IPTW
estimators to be unbiased, the probability of exposure for any given set
of covariates should be neither close to zero nor to 1. If a covariate
\emph{l} determines treatment, then this will be a practical positivity
violation as all individuals with a specific value of \(l_k\) will get
the treatment (ie.probability will be one) and all individuals with
other value(s) of \(l_k\) will not get the treatment (ie. probability
will be zero).

\section{Question 12: What are some potential advantages or
disadvantages to longitudinal
IPTW?}\label{question-12-what-are-some-potential-advantages-or-disadvantages-to-longitudinal-iptw}

\subsection{Advantages:}\label{advantages}

\begin{itemize}
\tightlist
\item
  Allows us to use confounded observational data to approximate data
  wherein the exposure was randomized.
\end{itemize}

\subsection{Disadvantages:}\label{disadvantages}

\begin{itemize}
\tightlist
\item
  It has high variance compared to alternative methods (it is
  inefficient).
\item
  It is easy to run into ositivity violations or near violations, which
  leadsto a biased estimator. Because the weights are the inverse of the
  product of the probabilities, if there are many time points even
  moderately small probabilities can end up producing quite large
  weights once everything is multiplied through. (Stablizing the weights
  does help.)
\item
  Relies on consistent estimation of the probabilities, often using
  parametric models. If model is misspecified, the parameter can be
  biased.
\end{itemize}


\end{document}
